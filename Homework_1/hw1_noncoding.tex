\documentclass{article}

% Language setting
% Replace `english' with e.g. `spanish' to change the document language
\usepackage{biblatex} %Imports biblatex package
\addbibresource{sample.bib}
\usepackage[english]{babel}
\usepackage{array}
\usepackage{amsmath}
\usepackage{pythonhighlight}
\newcolumntype{P}[1]{>{\centering\arraybackslash}p{#1}}
\newcolumntype{M}[1]{>{\centering\arraybackslash}m{#1}}

% Set page size and margins
% Replace `letterpaper' with `a4paper' for UK/EU standard size
\usepackage[letterpaper,top=2cm,bottom=2cm,left=3cm,right=3cm,marginparwidth=1.75cm]{geometry}

\usepackage{amsmath}
\usepackage{graphicx}
\usepackage[colorlinks=true, allcolors=blue]{hyperref}
\usepackage{setspace}
\usepackage{booktabs}
\usepackage[T1]{fontenc}
\usepackage{longtable}
\doublespacing

\begin{document}
\begin{titlepage}

\centering
\scshape
\vspace{\baselineskip}

%
\rule{\textwidth}{1.6pt}\vspace*{-\baselineskip}\vspace*{2pt}
\rule{\textwidth}{0.4pt}

{\Huge \textbf{\textsc{CS 450: Homework 1 \\
\vspace{15pt}}}}

\rule{\textwidth}{0.4pt}\vspace*{-\baselineskip}\vspace{3.2pt}
\rule{\textwidth}{1.6pt}\vspace{6pt}
\centerline{\textit{University of Illinois at Urbana-Champaign}} 
\vspace{1.5\baselineskip}


\large \centerline{\textbf{Author:} Nathan Glaser}
\large \centerline{\textbf{Net-ID:} nglaser3}
\quad

\large \centerline{\textbf{Professor:} Andreas Kloeckner}
\quad

\vfill
\large \centerline{August 28, 2024}
%
\pagenumbering{gobble}
\end{titlepage}

\tableofcontents
\newpage
\pagenumbering{arabic}

\section{Question 1}
To begin, the following equation must be satisfied (problem statement):
\begin{equation}
    f^{(n)}(x) = \frac{f(x) -2 \cdot f(x-h) + f(x-2h)}{h^2}
    \label{abc}
\end{equation}

The Taylor series expansion definition:
\begin{equation}
    \label{taylor_expansion_definition}
    \sum_{n=0}^\infty \frac{f^{(n)}(a)}{n!}(x-a)^n
\end{equation}

Expanding $f(x-h)$ using Eq. \ref{taylor_expansion_definition}, setting $x-a = h$:

\begin{equation}
    \label{fx_h}
    f(x-h) = f(x) - h\cdot f'(x) + h^2 \cdot \frac{f''(x)}{2}  ....
\end{equation}

Similarly, expanding $f(x-2h)$:
\begin{equation}
    \label{fx_2h}
    f(x-2h) = f(x) -2h\cdot f'(x) + 2h^2\cdot f''(x) ....
\end{equation}

plugging in the expanded forms of $f(x-h)$ and $f(x-2h)$ into Eq. \ref{abc}, we see which terms cancel out. 

$f(x)$ cancels out as $1-2+1 = 0$.

$f'(x)$ also cancels out, as $(-2 \cdot -1) + (1 \cdot-2) = 0$.

For $f''(x)$, $(-2 \cdot \frac{1}{2}) + (1 \cdot 2) = 1$.

Thus, this approximation is for the second derivative of $f(x)$.

\section{Question 2}
The truncation error is $\mathcal{O}(h^3)$, or third order accurate, as the lowest order term 'truncated' from the approximation is $\alpha h^3f'''(x)$. This approximation is exact when h is equal to 0.

\section{Question 3}
No Idea

\end{document}