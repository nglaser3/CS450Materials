\documentclass{article}

% Language setting
% Replace `english' with e.g. `spanish' to change the document language
\usepackage{biblatex} %Imports biblatex package
\addbibresource{sample.bib}
\usepackage[english]{babel}
\usepackage{array}
\usepackage{amsmath}
\usepackage{pythonhighlight}
\newcolumntype{P}[1]{>{\centering\arraybackslash}p{#1}}
\newcolumntype{M}[1]{>{\centering\arraybackslash}m{#1}}

% Set page size and margins
% Replace `letterpaper' with `a4paper' for UK/EU standard size
\usepackage[letterpaper,top=2cm,bottom=2cm,left=3cm,right=3cm,marginparwidth=1.75cm]{geometry}

\usepackage{amsmath}
\usepackage{graphicx}
\usepackage[colorlinks=true, allcolors=blue]{hyperref}
\usepackage{setspace}
\usepackage{booktabs}
\usepackage[T1]{fontenc}
\usepackage{longtable}
\doublespacing

\begin{document}
\begin{titlepage}

\centering
\scshape
\vspace{\baselineskip}

%
\rule{\textwidth}{1.6pt}\vspace*{-\baselineskip}\vspace*{2pt}
\rule{\textwidth}{0.4pt}

{\Huge \textbf{\textsc{CS 450: Homework 2 \\
\vspace{15pt}}}}

\rule{\textwidth}{0.4pt}\vspace*{-\baselineskip}\vspace{3.2pt}
\rule{\textwidth}{1.6pt}\vspace{6pt}
\centerline{\textit{University of Illinois at Urbana-Champaign}} 
\vspace{1.5\baselineskip}


\large \centerline{\textbf{Author:} Nathan Glaser}
\large \centerline{\textbf{Net-ID:} nglaser3}
\quad

\large \centerline{\textbf{Professor:} Andreas Kloeckner}
\quad

\vfill
\large \centerline{September 11, 2024}
%
\pagenumbering{gobble}
\end{titlepage}

\tableofcontents
\newpage
\pagenumbering{arabic}

\section*{Question 1}
\addcontentsline{toc}{section}{\protect\numberline{}Question 1}

\begin{equation}
    \Tilde{p}(r) = p(r) + \delta(r)
    \label{ptil_def}
\end{equation}

\begin{equation}
    \label{rdef}
    \Tilde{r} = r + \Delta r
\end{equation}

\begin{equation}
    \epsilon_{b,rel} = \frac{\left| \Tilde{p}-p \right|}{|p|} = \frac{\left| \delta \right|}{|p|}
    \label{brel_def}
\end{equation}

\begin{equation}
    \label{taydef}
    f(a) = \sum_{n=0}^\infty \frac{f^n(x)}{n!} (x-a)^n
\end{equation}


\begin{equation}
    \left| \delta'(r)\right| \leq 2\left| \Delta r \right|
    \label{del_prime}
\end{equation}

Expanding $\Tilde{p}(\Tilde{r})$ about r we obtain a definition for $\delta(r)$. 

First the preliminary expansion, ignoring second$+$ order terms:

\begin{equation}
    \Tilde{p}(\Tilde{r}) = \Tilde{p}(r) - \Tilde{p}'(r)\Delta r
\end{equation}

Substituting definitions of $\Tilde{p}$, the  L.H.S. is 0 and every instance of $\Tilde{p}$ is substituted with Eq. \ref{ptil_def}:

\begin{equation}
    0 = p(r) + \delta(r) - (p'(r)+\delta'(r))\Delta r 
\end{equation}

Next, Eq. \ref{del_prime}:

\begin{equation}
    0 \leq p(r) + \delta(r) - p'(r)\Delta(r)+2(\Delta r)^2
\end{equation}

Finally, cancelling the second order term and rearranging:

\begin{equation}
    \delta (r) \leq p'(r)\Delta r
\end{equation}

and then substituting into Eq. \ref{brel_def}:

\begin{equation}
    \epsilon_{b,rel} = \frac{\left| \delta \right|}{|p|} \leq \frac{p'(r)\Delta r}{p(r)} \cdot\frac{r}{r}
\end{equation}

\newpage


\section*{Question 2}
\addcontentsline{toc}{section}{\protect\numberline{}Question 2}

Inserting the polynomials into Eq. \ref{brel_def}, we obtain:
\begin{equation}
    \frac{|c^n|}{|(x-r)^n|}
    \label{q2_be}
\end{equation}

recognizing we are computing the p norm where p=1, and so the norm is simply the sum of coefficients. Further, This denominator is a binomial series of input x and r. So, the denominator becomes:
\begin{equation}
    |(x-r)^n| = \left|\sum_{i=0}^n {{r\choose x}r^{n-i}x^i} \right| = \sum_{i=0}^n {{r\choose x}r^{n-i}} \geq r^n
\end{equation}

Thus, because $r^n$ is less than the full series, and because the series is in the denominator of Eq. \ref{q2_be}, the relative backward error simplifies to:

\begin{equation}
    \leq \frac{|c^n|}{|(r)^n|}
\end{equation}



The root of $p(x)$ is simply r. The root of $\Tilde{p}(x)$ is found through inverting the function:
\begin{equation}
    root = r \pm c
    \label{root_pert}
\end{equation}

Where $\Tilde{p}$ has a root only at $r + c$ when n is odd, and $r \pm c$ when n is even. 

The relative forward error is simply Eq. a\ref{frel__}, and then from Eq. \ref{root_pert}:

\begin{equation}
    \epsilon_{f,rel} = \frac{|\Tilde{r}-r|}{|r|}
    \label{frel__}
\end{equation}



\begin{equation}
    \epsilon_{f,rel} = \frac{|c|}{|r|}
\end{equation}

Finally, as n increases the problem becomes worse and worse conditioned. 
This can be seen blatantly from the definition of the conditioning number:

\begin{equation}
    \left|\frac{c}{r}\right| \leq \kappa \left|\frac{c}{r}\right|^n
\end{equation}
and solving for kappa, we see that, because $r\>c$, as n increases so does kappa:

\begin{equation}
    \left| \frac{r}{c}\right|^{n-1} \leq \kappa
\end{equation}
And because kappa is increasing, the problem is worse and worse conditioned.



\section*{Question 3}
\addcontentsline{toc}{section}{\protect\numberline{}Question 3}

To begin, because f is diffeomorphic (f is differentiable, f has an inverse, f's inverse is differentiable), the following theorem applies: 

\begin{equation}
    \left[f^{-1}\right]'(x) = \frac{1}{f'(f^{-1}(x))}
\end{equation}

utilizing this definition, as well as the definition of $\kappa_{abs}$ below:

\begin{equation}
    \kappa_{abs} = \left| f'(x) \right|
\end{equation}

we determine that:

\begin{equation}
    \kappa_{inv} = \left| \frac{1}{p'(p^{-1}(x))} \right|
\end{equation}

recognizing that both $p^{-1}(x)$ and $p'(x)$ are defined such that their domains are x $\in \Re$, then the above denominator is identical to saying $p'(x)$, and again using the definition for $\kappa_{abs}$ from above:

\begin{equation}
    \kappa_{inv} = \frac{1}{p'(x)} = \frac{1}{\kappa_{abs}}
\end{equation}

\quad

\quad

Now for the next part. To recap:
\begin{equation}
    \epsilon_{b,rel} \geq \left|\frac{\Delta r}{r}\right|\left| \frac{r p'(r)}{p(r)}\right|
    \label{q1end}
\end{equation}

\begin{equation}
    \epsilon_{f,rel} = \left|\frac{\Delta r}{r}\right|
    \label{epsfrel}
\end{equation}

\begin{equation}
    \epsilon_{b,rel} = \left|\frac{\Tilde{p} - p}{p}\right|
    \label{epsbrel}
\end{equation}

\begin{equation}
    \kappa_{rel} = \left| \frac{r p'(r)}{p(r)}\right|
    \label{kapreldef}
\end{equation}

\begin{equation}
    \epsilon_{f,rel} \leq \kappa_{rel} \epsilon_{b,rel}
    \label{kappadef}
\end{equation}

\begin{equation}
    \kappa_{inv}=\frac{1}{\kappa_{abs}}=\frac{r}{\kappa_{rel} p}
    \label{kappainvdef}
\end{equation}

To begin, solving Eq. \ref{kappadef} for $\kappa$:
\begin{equation}
    \kappa_{rel} \geq \frac{\epsilon_{f,rel}}{\epsilon_{b,rel}}
\end{equation}

And then inserting Eqs. \ref{epsfrel} and \ref{kapreldef} into Eq. \ref{q1end}:
\begin{equation}
    \epsilon_{b,rel} \geq \epsilon_{f,rel} \kappa_{rel}
\end{equation}

solving for $\kappa$:
\begin{equation}
    \kappa_{rel} \leq \frac{\epsilon_{b,rel}}{\epsilon_{f,rel}}  
\end{equation}

Solving Eq. \ref{kappainvdef} for $\kappa_{rel}$:
\begin{equation}
    \kappa_{rel} = \frac{r}{\kappa_{inv} p}
\end{equation}

Inserting this into the previous equation, multiplying both sides by p and r (to convert the relative forward and backward errors to absolute):

\begin{equation}
    \frac{1}{\kappa_{inv}} \leq \frac{\epsilon_{b,abs}}{\epsilon_{f,abs}}
\end{equation}

Finally, recognizing the R.H.S is simply $\frac{1}{\kappa_{abs}}$, and taking the inverse of the equation:

\begin{equation}
    \kappa_{inv} \geq \kappa_{abs}
\end{equation}

\end{document}