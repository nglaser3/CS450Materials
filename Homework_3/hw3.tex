\documentclass{article}

% Language setting
% Replace `english' with e.g. `spanish' to change the document language
\usepackage{biblatex} %Imports biblatex package
\addbibresource{sample.bib}
\usepackage[english]{babel}
\usepackage{array}
\usepackage{amsmath}
\usepackage{pythonhighlight}
\newcolumntype{P}[1]{>{\centering\arraybackslash}p{#1}}
\newcolumntype{M}[1]{>{\centering\arraybackslash}m{#1}}

% Set page size and margins
% Replace `letterpaper' with `a4paper' for UK/EU standard size
\usepackage[letterpaper,top=2cm,bottom=2cm,left=3cm,right=3cm,marginparwidth=1.75cm]{geometry}

\usepackage{amsmath}
\usepackage{graphicx}
\usepackage[colorlinks=true, allcolors=blue]{hyperref}
\usepackage{setspace}
\usepackage{booktabs}
\usepackage[T1]{fontenc}
\usepackage{longtable}
\doublespacing

\begin{document}
\newcommand{\R}{${\rm I\!R}$}
\newcommand{\norm}[1]{\| #1 \|_2}
\newcommand{\nmv}[2]{\norm{\textbf{#1}\Vec{#2}}}
\newcommand{\abs}[1]{\bigl| #1 \bigr|}
\begin{titlepage}

\centering
\scshape
\vspace{\baselineskip}

%
\rule{\textwidth}{1.6pt}\vspace*{-\baselineskip}\vspace*{2pt}
\rule{\textwidth}{0.4pt}

{\Huge \textbf{\textsc{CS 450: Homework 3 \\
\vspace{15pt}}}}

\rule{\textwidth}{0.4pt}\vspace*{-\baselineskip}\vspace{3.2pt}
\rule{\textwidth}{1.6pt}\vspace{6pt}
\centerline{\textit{University of Illinois at Urbana-Champaign}} 
\vspace{1.5\baselineskip}


\large \centerline{\textbf{Author:} Nathan Glaser}
\large \centerline{\textbf{Net-ID:} nglaser3}
\quad

\large \centerline{\textbf{Professor:} Andreas Kloeckner}
\quad

\vfill
\large \centerline{September 25, 2024}
%
\pagenumbering{gobble}
\end{titlepage}

\tableofcontents
\newpage
\pagenumbering{arabic}

\section*{Question 1}
\addcontentsline{toc}{section}{\protect\numberline{}Question 1}

To begin, some prerequisite definitions:

\begin{equation}
    \label{eq:invdef}
    \textbf{A}^{-1}\textbf{A}\Vec{x} = \Vec{x}
\end{equation}

\begin{equation}
    \norm{\textbf{A}\Vec{x}} \leq \norm{\textbf{A}}\cdot\norm{\Vec{x}}
    \label{eq:leqdef}
\end{equation}

To begin, we take the 2-norm of Eq. \ref{eq:invdef}:

\begin{equation}
    \norm{\textbf{A}^{-1}\textbf{A}\Vec{x}} = \norm{\Vec{x}}
\end{equation}

substituting in $\textbf{A}^{-1}$ in for $\textbf{A}$ and $\textbf{A}\Vec{x}$ for $\Vec{x}$ in Eq. \ref{eq:leqdef}, we obtain the following inequality:

\begin{equation}
    \norm{\Vec{x}} \leq \norm{\textbf{A}^{-1}}\norm{\textbf{A}\Vec{x}}
\end{equation}

and simply rearranging:

\begin{equation}
    \frac{\norm{\Vec{x}}}{\norm{\textbf{A}^{-1}}} \leq \norm{\textbf{A}\Vec{x}}
    \label{eq:ansq1}
\end{equation}

\newpage
\section*{Question 2}
\addcontentsline{toc}{section}{\protect\numberline{}Question 2}

To begin, add and subtract $\nmv{B}{x}$ from some arbitrary norm $\nmv{A}{x}$ and apply a total absolute value to the equation:

\begin{equation}
    \abs{\nmv{A}{x}} = \abs{\nmv{A}{x} + \nmv{B}{x} - \nmv{B}{x}}
\end{equation}

Then applying the triangle-rule, pulling out $\nmv{A}{x}$ and $-\nmv{B}{x}$:

\begin{equation}
    \nmv{A}{x} \leq \abs{\nmv{A}{x}-\nmv{B}{x}} + \nmv{B}{x}
    \label{eq:midq2}
\end{equation}

note the absolute value bars dropped from the two standalone norms, this is because a norm is always positive. Next, the 'inverse' triangle rule (Page 53 in textbook) is:

\begin{equation}
    \abs{\norm{\Vec{x}} - \norm{\Vec{y}}} \leq \norm{\Vec{x} - \Vec{y}}
\end{equation}

applying this to Eq. \ref{eq:midq2}, we obtain:

\begin{equation}
    \nmv{A}{x} \leq \abs{\norm{\textbf{A}\Vec{x}-\textbf{B}\Vec{x}}} + \nmv{B}{x}
\end{equation}

And then pulling out $\Vec{x}$ from the large chunk, and then because A-B linearly scales vector $\Vec{x}$, we can pull that out of the norm (maintaining a norm on it):

\begin{equation}
    \nmv{A}{x} \leq \norm{\textbf{A}-\textbf{B}}\cdot\norm{\Vec{x}} + \nmv{B}{x}
    \label{eq:ansq2}
\end{equation}


\newpage
\section*{Question 3}
\addcontentsline{toc}{section}{\protect\numberline{}Question 3}

To begin, we assume the following condition to be true:

\begin{equation}
    \frac{\norm{\textbf{A}-\textbf{B}}}{\norm{\textbf{A}}} < \frac{1}{\kappa_2 (\textbf{A})}
\end{equation}

If we multiply the L.H.S. by $\frac{\norm{\Vec{x}}}{\norm{\Vec{x}}}$ (1):

\begin{equation}
    \frac{\norm{\textbf{A}-\textbf{B}}}{\norm{\textbf{A}}} \cdot \frac{\norm{\Vec{x}}}{\norm{\Vec{x}}} < \frac{1}{\kappa_2 (\textbf{A})}
\end{equation}

And then applying a similair method used at the end of the previous question:

\begin{equation}
    \frac{\nmv{A}{x} - \nmv{B}{x}}{\norm{\textbf{A}}\norm{\Vec{x}}} < \frac{1}{\kappa_2 (\textbf{A})}
\end{equation}

And because we know A is invertible / non-singular, we can utilize Eq. \ref{eq:ansq1}:

\begin{equation}
    \frac{\frac{\norm{\Vec{x}}}{\norm{\textbf{A}^{-1}}} - \nmv{B}{x}}{\norm{\textbf{A}}\norm{\Vec{x}}} < \frac{1}{\kappa_2 (\textbf{A})}
\end{equation}

substituting in the definiton for a matrix conditioning number, $\kappa = \norm{\textbf{A}^{-1}}\norm{\textbf{A}}$, and rearranging:

\begin{equation}
    \frac{\norm{\Vec{x}}}{\norm{\textbf{A}^{-1}}} - \nmv{B}{x} < \frac{\norm{\Vec{x}}}{\norm{\textbf{A}^{-1}}}
\end{equation}

and further, subtracting out the fraction:

\begin{equation}
    \nmv{B}{x} > 0
\end{equation}

Because the norm of $\textbf{B}\Vec{x}$ is non-zero, we know it is non-zero. Assuming that $\textbf{B}\in \R^{n \times n}$ and that $\Vec{x}$ is a non-zero vector, then we know that $\textbf{B}$'s Null space must only contain the null vector, and is therefor not-singular. 
\end{document}